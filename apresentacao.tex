% Aula 04 - Apresentações

% ------Falta terminar a anotação------

% beamer é uma classe usada para fazer slides
% \usetheme{nome-do-tema}
% \usecolortheme{nome-da-cor}
\documentclass{beamer}
\usetheme{default}

\usepackage[utf8]{inputenc}
\usepackage[portuguese]{babel}
\usepackage{xcolor} % Pacote para definir novas cores

\title{Minha primeira apresentação}
\institute{Universidade Federal do Delta do Parnaíba}
\author{Francisco Alves Ribeiro Neto}
% \date{00/00/0000}
\logo{\includegraphics[scale=0.12]{imagens/ufdpar-logo.png}}
% Esse logo fica em todas as páginas

\begin{document}

\frame{\titlepage}

\frame{\tableofcontents}

\section{UsandoBeamer}

\frame{
    \frametitle{Características}
    \begin{itemize}
        \item<1->Classe
              \begin{block}{Título do bloco}
                  Texto dentro do bloco
              \end{block}

              \begin{alertblock}{Título do bloco}
                  Texto dentro do bloco de alerta
              \end{alertblock}

              \begin{exampleblock}{Título do bloco}
                  Texto dentro do bloco de exemplo
              \end{exampleblock}

              % Definindo novas cores
              \definecolor{cor1}{rgb}{0.12,0.22,0.55}
              \definecolor{cor2}{rgb}{0.25,0.40,0.83}
              \setbeamercolor{cor1}{bg=cor1, fg=cor2}
              \setbeamercolor{cor2}{bg=cor2, fg=cor1}

              % Usando beamercolorbox com nova cor
              \begin{beamercolorbox}[shadow=true,rounded=true,center]{cor1}
                  Texto dentro da caixa 1
              \end{beamercolorbox}

              \vspace{0.3cm}

              % Usando beamerboxesrounded com as cores definidas
              \begin{beamerboxesrounded}[lower=cor2, upper=cor1, shadow=true]{Título da caixa}
                  Texto dentro da caixa 2
              \end{beamerboxesrounded}

        \item<2->Sobreposições
        \item<3->Transições
    \end{itemize}
}

\subsection{Uma subseção}
\frame{
    \frametitle{Subseção}
    Assim se divide uma apresentação em sessões.
}

\end{document}
