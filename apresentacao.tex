% Aula 04 - Apresentações

% beamer é uma classe usada para fazer slides
\documentclass{beamer}

\usetheme{default}
% \usecolortheme{nome-da-cor}

\usepackage[utf8]{inputenc}
\usepackage[portuguese]{babel}
\usepackage{xcolor} % Pacote para definir novas cores

% Definição de informações sobre a apresentação
% Essas informações são usadas principalmente no slide de título e para personalizar a apresentação.
\title{Minha primeira apresentação}
\institute{Universidade Federal do Delta do Parnaíba}
\author{Francisco Alves Ribeiro Neto}
% \date{00/00/0000}
\logo{\includegraphics[scale=0.12]{imagens/ufdpar-logo.png}}
% Este logo aparecerá em todas as páginas

\begin{document}

% Slide inicial com o título da apresentação
\frame{\titlepage}

% Slide de conteúdo com a tabela de conteúdo (índice)
\frame{\tableofcontents}

\section{Seção}

% Efeitos de sobreposição: cada item aparecerá em ordem conforme o avanço da apresentação
% <n-> indica que o item aparece a partir da transição n
\frame{
    \frametitle{Características} 
    \begin{itemize}
        \item<1->Item 1
        \item<2->Classe
        \item<3->Sobreposições
        \item<4->Transições
    \end{itemize}
}

% Slide separado com os blocos
\frame{
    \frametitle{Blocos de Destaque}
    \begin{block}{Título do bloco}
        Texto dentro do bloco
    \end{block}

    \begin{alertblock}{Título do bloco}
        Texto dentro do bloco de alerta
    \end{alertblock}

    \begin{exampleblock}{Título do bloco}
        Texto dentro do bloco de exemplo
    \end{exampleblock}
}

% Slide separado com as caixas coloridas
\frame{
    \frametitle{Caixas Coloridas}

    % Definindo novas cores
    \definecolor{cor1}{rgb}{0.12,0.22,0.55}
    \definecolor{cor2}{rgb}{0.25,0.40,0.83}
    \setbeamercolor{cor1}{bg=cor1, fg=cor2}
    \setbeamercolor{cor2}{bg=cor2, fg=cor1}

    % Usando beamercolorbox com nova cor
    \begin{beamercolorbox}[shadow=true,rounded=true,center]{cor1}
        Texto dentro da caixa 1
    \end{beamercolorbox}

    \vspace{0.3cm}

    % Usando beamerboxesrounded com as cores definidas para uma caixa arredondada
    \begin{beamerboxesrounded}[lower=cor2, upper=cor1, shadow=true]{Título da caixa}
        Texto dentro da caixa 2
    \end{beamerboxesrounded}
}

\subsection{Uma subseção}
\frame{
    \frametitle{Subseção}
    Assim se divide uma apresentação em sessões.
}

\end{document}