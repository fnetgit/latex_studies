% Aula 02 - Matemática

\documentclass[12pt]{article}
\usepackage[utf8]{inputenc}
\usepackage{amsmath}

\begin{document}

\begin{itemize}
    \item Operações básicas: \\
          $x + y = 3$ \\
          $x \cdot y = 5$ \\
          $x * y = 5$ \\
          $\frac{x}{y} = 2$

          \medskip

    \item Potenciação:
          \[
              x^2 + y^2 = 10
          \]
          \[
              z^{x+y}
          \]
          \[
              x^{n^j}
          \]

          \medskip

    \item Subíndice: \\
          $x_i$ \\
          $x_{i+1}$

          \medskip

    \item Fração: \\
          $\frac{a}{b}$ \\
          $\frac{x+1}{y+2}$

          \medskip

    \item Raiz quadrada: \\
          $\sqrt{x}$ \\
          $\sqrt{x+3}$

          \medskip

    \item Alguns Símbolos:
          \[
              \begin{array}{c}
                  \alpha  \\
                  \Omega  \\
                  \mu     \\
                  \vec{a} \\
                  \Delta  \\
                  \approx \\
                  \neq    \\
                  \sum    \\
                  \sum_{i=0}^{N}
              \end{array}
          \]

          \medskip

    \item Matrizes: \\
          A)
          \[
              \begin{pmatrix}
                  a & b & c \\
                  d & e & f
              \end{pmatrix}
          \]

          \medskip

          B)
          \[
              \begin{bmatrix}
                  a & b & c \\
                  d & e & f
              \end{bmatrix}
          \]
\end{itemize}

\section*{Lista de Exercícios - 01}

\begin{enumerate}

    \item Calcule o determinante da seguinte matriz $A =
              \begin{bmatrix}
                  4  & -4 \\
                  -6 & 16
              \end{bmatrix}$.

    \item Calcule a seguinte integral:
          \[
              \int_0^1 \left( x^3 + x^2 + x + \sqrt[5]{x+1} \right) \, dx
          \]

    \item Determine se a seguinte série é convergente.
          \[
              \sum_{k=1}^{\infty} \frac{1}{k^2}
          \]

\end{enumerate}

\end{document}
