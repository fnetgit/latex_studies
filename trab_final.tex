\documentclass[12pt,a4paper]{article}
\usepackage[top=3cm,right=2cm,bottom=2cm,left=3cm]{geometry}
\usepackage[utf8]{inputenc}
\usepackage{amsmath,amssymb}
\usepackage{setspace}
\usepackage{helvet}
\renewcommand{\familydefault}{\sfdefault}
\usepackage[brazil]{babel}

\begin{document}

\begin{titlepage}
  \begin{center}
    \onehalfspacing

    \textbf{UNIVERSIDADE FEDERAL DO DELTA DO PARNAÍBA} \\
    Núcleo de Estudos em Matemática - NEMAT

    \vspace{4cm}
    \textbf{Francisco Alves Ribeiro Neto}

    \vspace{4cm}
    \textbf{\LARGE LISTA DE EXERCÍCIOS - LOGARITMOS}

    \vfill
    Parnaíba - PI \\
    2025
  \end{center}
\end{titlepage}

\onehalfspacing
\section*{Exercícios sobre Logaritmos}

\subsection*{Conceitos e histórico}
\begin{enumerate}
  \item O que é um logaritmo?

  \item Assinale a alternativa que expressa corretamente o significado da igualdade $\log_b(a) = c$:
        \begin{enumerate}
          \item $a^b = c$
          \item $b^c = a$
          \item $c^a = b$
          \item $b^a = c$
        \end{enumerate}

  \item Sobre a história dos logaritmos, julgue as afirmações a seguir como verdadeiras (V) ou falsas (F), e assinale a alternativa correta:

        \begin{enumerate}
          \item[1.] (~~) O logaritmo foi criado para facilitar cálculos manuais complexos, como multiplicações e divisões.
          \item[2.] (~~) John Napier é considerado o criador do conceito de logaritmos no início do século XVII.
          \item[3.] (~~) Os logaritmos naturais têm base 10 e foram os primeiros a serem utilizados na engenharia.
          \item[4.] (~~) Logaritmo neperiano se refere ao logaritmo de base \(e\), que é uma constante irracional aproximadamente igual a 2{,}71828.
        \end{enumerate}

        \textbf{Alternativas:}
        \begin{enumerate}
          \item[(a)] V, V, F, V
          \item[(b)] F, F, V, V
          \item[(c)] V, F, V, F
          \item[(d)] F, V, V, F
        \end{enumerate}

\end{enumerate}

\subsection*{Condições de existência e propriedades operatórias}
\begin{enumerate}
  \setcounter{enumi}{3}
  \item Qual das expressões abaixo está bem definida?
        \begin{enumerate}
          \item $\log_{-2}(8)$
          \item $\log_{2}(-8)$
          \item $\log_{0}(8)$
          \item $\log_{2}(8)$
        \end{enumerate}

  \item Sabendo que $\log 2 = 0{,}3010$ e $\log 3 = 0{,}4771$, qual o valor de $\log(18)$?
        \begin{enumerate}
          \item[(a)] 0{,}7781
          \item[(b)] 0{,}9542
          \item[(c)] 1{,}2552
          \item[(d)] 1{,}2300
        \end{enumerate}

  \item Simplifique a expressão: $\log_3(81)$.
\end{enumerate}

\subsection*{Funções e equações logarítmicas}
\begin{enumerate}
  \setcounter{enumi}{6}
  \item O gráfico da função $f(x) = \log_2(x)$ é crescente ou decrescente?

  \item Resolva a equação: $\log_2(x) = 5$

  \item Qual o domínio da função $f(x) = \log(x - 2)$?

  \item Qual é a função inversa de $f(x) = \log_2(x)$?
\end{enumerate}

\newpage

\section*{Gabarito}

\begin{enumerate}
  \item O logaritmo é o inverso da exponenciação. Ele responde à pergunta: "A que potência devemos elevar a base para obter um determinado número?". Ou seja, $\log_b(a) = c$ significa que a base $b$ elevada à potência $c$ resulta no número $a$. Isso é, $b^c = a$.

  \item (b) A igualdade $\log_b(a) = c$ significa que $b^c = a$.

  \item (a)
        \begin{itemize}
          \item (V) Criado para facilitar cálculos manuais.
          \item (V) Napier foi o primeiro a publicar o conceito de logaritmos.
          \item (F) Logaritmos naturais têm base \(e\), não 10 (essa base 10 é o logaritmo decimal).
          \item (V) “Neperiano” designa exatamente o logaritmo de base \(e\).
        \end{itemize}


  \item (d) A expressão $\log_2(8)$ está bem definida, pois a base é 2 (positiva e diferente de 1), e o argumento é 8 (positivo).
        Para que o logaritmo seja bem definido, a base deve ser positiva e diferente de 1, e o argumento deve ser maior que 0.

  \item (c) Como $18 = 2 \cdot 3^2$, temos:
        \[
          \log(18) = \log(2) + \log(3^2) = \log(2) + 2 \cdot \log(3)
        \]
        \[
          \log(18) = 0{,}3010 + 2 \cdot 0{,}4771 = 0{,}3010 + 0{,}9542 = 1{,}2552
        \]

  \item Para simplificar $\log_3(81)$, sabemos que $81 = 3^4$, então:
        \[
          \log_3(81) = \log_3(3^4) = 4.
        \]
        Portanto, $\log_3(81) = 4$.

  \item O gráfico da função $f(x) = \log_2(x)$ é crescente, pois, à medida que $x$ aumenta, o valor de $f(x)$ também aumenta.

  \item Para resolver a equação $\log_2(x) = 5$, devemos reescrever a equação na forma exponencial:
        \[
          x = 2^5.
        \]
        Logo, $x = 32$.

  \item O domínio da função $f(x) = \log(x - 2)$ é determinado pela condição de que o argumento do logaritmo deve ser positivo:
        \[
          x - 2 > 0 \quad \Rightarrow \quad x > 2.
        \]
        Portanto, o domínio da função é $x > 2$.

  \item A função inversa de $f(x) = \log_2(x)$ é obtida trocando as variáveis e resolvendo para $x$:
        \[
          y = \log_2(x) \quad \Rightarrow \quad x = 2^y.
        \]
        Logo, a função inversa é $f^{-1}(x) = 2^x$.
\end{enumerate}

\end{document}
