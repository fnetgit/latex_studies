\documentclass[12pt]{article}
\usepackage[utf8]{inputenc}
\usepackage[hyphens]{url}
\usepackage[colorlinks=true,urlcolor=blue]{hyperref}

\begin{document}

Olá, Mundo!  

Francisco\\
testes matemáticos: $\sqrt{2}, \sqrt[2]{23}$

\newpage

Segunda Página\hspace{2cm}escrevendo um texto

\begin{equation}
x^2 - 1 = 0
\end{equation}

\noindent\textsc{O \textbf{Teorema de Pitágoras} é uma relação entre as medidas dos lados
de um triângulo retângulo. De acordo com esse teorema, o quadrado da medida
da hipotenusa é igual à soma dos quadrados das medidas dos catetos.
Ou seja, se os catetos medem \textit{a} e \textit{b} e a hipotenusa mede \textit{c}, então}
\begin{equation}
    c^2 = a^2 + b^2.
\end{equation}

\noindent\textsc{Podemos utilizar essa relação para obter a medida de um
dos lados do triângulo retângulo, supondo que os outros dois sejam conhecidos.}

Veja mais sobre "Teorema de Pitágoras" em: \url{https://brasilescola.uol.com.br/matematica/teorema-pitagoras.html}

\end{document}
