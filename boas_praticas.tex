\documentclass[a4paper,12pt]{article}

% ======================
% Pacotes essenciais
% ======================
\usepackage[utf8]{inputenc}    % Suporte a acentuação
\usepackage[brazil]{babel}     % Idioma português
\usepackage[T1]{fontenc}       % Codificação correta das fontes
\usepackage{amsmath, amssymb}  % Matemática
\usepackage{graphicx}          % Inserção de imagens
\usepackage{xcolor}            % Cores (para o código-fonte)
\usepackage{hyperref}          % Pacote para links e URLs interativos no documento
\hypersetup{
  colorlinks=true,      % Ativa links coloridos ao invés de caixas ao redor
  linkcolor=black,      % Define a cor dos links internos (como no sumário)
  citecolor=blue,       % Cor dos links para citações bibliográficas
  urlcolor=blue         % Cor dos links para URLs
}

\usepackage{geometry}          % Margens personalizadas
\geometry{margin=2.5cm}

% ======================
% Título
% ======================
\title{Guia Básico de LaTeX com Boas Práticas}
\author{Escrito em \LaTeX}
\date{\today}

\begin{document}

\maketitle
\tableofcontents
\newpage

% ======================
\section{Introdução}
Este documento ensina LaTeX usando boas práticas e exemplos reais. Ideal para quem está começando e deseja manter os arquivos limpos e organizados.

% ======================
\section{Estrutura Básica}
Todo documento começa com a classe e os pacotes:

\begin{verbatim}
\documentclass[a4paper,12pt]{article}
\usepackage[utf8]{inputenc}
\usepackage{graphicx}
\begin{document}
Conteúdo aqui
\end{document}
\end{verbatim}

% ======================
\section{Seções e Organização}
\begin{verbatim}
\section{Título}
\subsection{Subseção}
\subsubsection{Subsubseção}
\end{verbatim}

% ======================
\section{Texto Formatado}
Você pode destacar partes do texto usando comandos simples. Veja como:

\begin{itemize}
  \item Negrito: use \verb|\textbf{texto}| → \textbf{assim}
  \item Itálico: use \verb|\textit{texto}| → \textit{assim}
  \item Sublinhado: use \verb|\underline{texto}| → \underline{assim}
  \item Para mostrar um código curto no meio da frase, use \verb|\verb|, como em \verb|\LaTeX|
\end{itemize}

% ======================
\section{Listas}

\subsection*{Não ordenada}
\begin{verbatim}
\begin{itemize}
  \item Item 1
  \item Item 2
\end{itemize}
\end{verbatim}

\subsection*{Ordenada}
\begin{verbatim}
\begin{enumerate}
  \item Primeiro
  \item Segundo
\end{enumerate}
\end{verbatim}

% ======================
\section{Equações}

\subsection*{No meio do texto}
Para colocar uma equação dentro de uma linha de texto, use cifrões simples \verb|$...$| ou o comando \verb|\(...\)|.  
Exemplo: \verb|E = mc^2| → \( E = mc^2 \)

\subsection*{Equação Centralizada}
Para destacar uma equação no centro da página, use o ambiente \texttt{equation}:

\begin{verbatim}
\begin{equation}
  E = mc^2
\end{equation}
\end{verbatim}

Resultado:
\begin{equation}
  E = mc^2
\end{equation}

% ======================
\section{Imagens}
\begin{verbatim}
\begin{figure}[h!]
  \centering
  \includegraphics[width=0.5\textwidth]{exemplo.png}
  \caption{Exemplo de imagem}
  \label{fig:imagem}
\end{figure}
\end{verbatim}

% ======================
\section{Citações e Referências}

\subsection*{Manual}
\begin{verbatim}
\begin{thebibliography}{9}
\bibitem{knuth}
Knuth, Donald.
\textit{The TeXbook}.
Addison-Wesley, 1984.
\end{thebibliography}
\end{verbatim}

\subsection*{BibTeX}
\begin{verbatim}
\bibliographystyle{abbrv}
\bibliography{referencias}
\end{verbatim}

% ======================
\section{Boas Práticas Recomendadas}
\begin{itemize}
  \item Separe o conteúdo em arquivos com \verb|\input{}|
  \item Organize imagens em uma pasta \texttt{img/}
  \item Evite usar \verb|\\| para pular linhas
  \item Crie comandos próprios com \verb|\newcommand{}|
  \item Use UTF-8 e compile com \texttt{pdflatex} ou \texttt{lualatex}
  \item Use \texttt{.bib} para referências bibliográficas
\end{itemize}

% ======================
\section{Conclusão}
O \LaTeX\ é poderoso e ideal para textos técnicos e científicos. Seguindo boas práticas, você terá documentos bem organizados, limpos e profissionais.

\end{document}
